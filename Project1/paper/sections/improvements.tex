The code has been built since the beginning as a fully object-oriented program, defining thus a clear and well organized structure suitable also for further applications to other systems. New sub-classes can in principle be added to the primary ones, allowing the user to approach other problems in the context of Quantum Mechanics. \textcolor{red}{The speed of the code can also be enhanced by the adoption of libraries pointed towards scientific computing (e.g. \texttt{<armadillo>}), but this would probably require heavy modifications to the already implemented structure.} 

A finer research of the $\alpha$ value which minimizes the energy of the system could be achieved substituting the very rough standard gradient descent with an improved version of it (e.g. stochastic gradient or conjugate gradient), maybe also exploiting some built-in python functions to approximate the Hessian matrix needed for the implementation. 
