A Bose-Einstein condensate is a state of matter that we can observe in a system of bosons confined and cooled down to temperatures close to the absolute zero. Since this physical concept was first proposed by Einstein and Bose \cite{einstein_original} in the 1920s, a lot of years have passed until the first experimental observation of this phenomenon, which was achieved by C. E. Wieman \& E. A. Cornell \cite{wieman} and W. Ketterle \cite{ketterle} only in 1995. Working independently, these two groups were able to observe a system of bosons constituted respectively by Rubidium atoms and Sodium atoms trapped at extremely low temperatures. 

Nowadays these systems are still intensively studied both from a experimental and theoretical perspective. In the former case, a BEC is produced in the laboratory exploiting laser cooling techniques and magnetic traps to confine the bosons \cite{lasercooling}. A theoretical analysis can instead be performed adopting a statistical approach, especially when the density of the system is sufficiently high to prevent from a proper description through the Gross-Pitaevskii equation \cite{gross}\cite{pita}. The problem can then be faced with the implementation of a Variational Monte Carlo (VMC) code: it allows to get pieces of information about a system using a statistical approach instead of trying to derive analytical solutions. Exploiting the concept of Markov chains \cite{markov} and algorithms based on pseudo-random numbers generation, we are allowed to explore all the possible states of a system and thus obtain numerical quantities related to it as averages over the possible configurations. The Monte Carlo method becomes particularly effective and time-saving when we have to deal with multidimensional integrals whose analytical evaluation would be impossible and the usage of traditional numerical methods would require a great effort. \\


In this project we are going to analyze the properties of a system constituted by $^{87}$Rb atoms through the implementation of a VMC code. We will consider a non-interacting hamiltonian and then switch to the interacting case, adopting for both the configurations a properly built trial wavefunction. The traps for the bosons will be described through a spherical or an elliptical potential. In our analysis we included also the search for the value of the variational parameter $\alpha$ corresponding to the minimum energy of the system. The VMC steps were performed exploiting different variants of the Metropolis algorithm. \cite{metropolis} \cite{hastings} (\textit{Add here missing parts about resampling methods and parallelization}).


\subsection{OVERVIEW OF THE WORKFLOW}
\textit{(We'll write this section at the end of the work, when everything will be defined)}