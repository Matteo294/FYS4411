A Bose-Einstein condensate (BEC) is a state of matter that we can observe in a system of bosons confined and cooled down to temperatures close to the absolute zero. Since this physical concept was first proposed by Einstein and Bose \cite{einstein_original} in the 1920s, a lot of years have passed until the first experimental observation of this phenomenon, which was achieved by C. E. Wieman \& E. A. Cornell \cite{wieman} and W. Ketterle \cite{ketterle} only in 1995. Working independently, these two groups were able to observe a system of bosons constituted respectively by Rubidium atoms and Sodium atoms trapped at extremely low temperatures. 

Nowadays these systems are still intensively studied both from a experimental and theoretical perspective. In the former case, a BEC is produced in the laboratory exploiting laser cooling techniques and magnetic traps to confine the bosons \cite{lasercooling}. A theoretical analysis can instead be performed adopting a statistical approach, especially when the density of the system is sufficiently high to prevent from a proper description through the Gross-Pitaevskii equation \cite{gross}\cite{pita}. The problem can then be faced with the implementation of a Variational Monte Carlo (VMC) code, which allows to get pieces of information about a system using a statistical approach instead of trying to derive analytical solutions. Exploiting the concept of Markov chains \cite{markov} and algorithms based on pseudo-random numbers generation, we are allowed to explore all the possible states of a system and thus obtain numerical quantities related to it as averages over the possible configurations. The Monte Carlo method becomes particularly effective and time-saving when we have to deal with multidimensional integrals whose analytical evaluation would be impossible and the usage of traditional numerical methods would require a great effort. \\


In this project we will analyzed the properties of a system constituted by $^{87}$Rb atoms through the implementation of a VMC code. We considered a non-interacting Hamiltonian and then switched to the interacting case, adopting for both the configurations a properly built trial wavefunction. The traps for the bosons were described through a spherical or an elliptical potential. In our analysis we included also the search for the value of the variational parameter $\alpha$ corresponding to the minimum energy of the system. The VMC steps were performed exploiting different variants of the Metropolis algorithm \cite{metropolis} \cite{hastings} and a statistical analysis of the obtained results was provided by the blocking technique \cite{Marius}. The parallelization of the code through \texttt{OpenMP} contributed to a reduction of the execution time.


\subsection{OVERVIEW OF THE WORKFLOW}
In the following section we will introduce the physical aspects behind the considered problem, together with the mathematical instruments that we adopted for the description of the system in terms of Hamiltonians, wavefunctions and one-body densities. A first mention to the error analysis is also reported, with particular attention to the problem of correlated data. In Section \ref{sec:methods} we will switch to a description of the methods adopted to treat the problem, with particular attention to the different variant of the Metropolis algorithm, the gradient descent technique and the blocking method for the error analysis. The code structure and implementation is then treated in Section \ref{sec:code}, followed by the presentation of the obtained results in Section \ref{sec:results} and the subsequent discussion in Section \ref{sec:discussion}. We will proceed specifying possible further improvements to enhance code performance and some more detailed analysis that could be included (Section \ref{sec:improvements}), concluding then with the final remarks about the project (Section \ref{sec:conclusions}). Some additional results and calculations can be found in the Appendix.