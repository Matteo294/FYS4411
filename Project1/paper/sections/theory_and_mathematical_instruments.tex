When considering a system of particles at a high temperature and in the limit of low particle density, the Maxwell-Boltzmann distribution provides a very good description of the statistical features of the system. However, when the temperature decreases, approaching the absolute zero, a classical description is no more sufficient, giving way to a Quantum mechanical treatment. In this context, the bosonic or fermionic nature of the particles comes into play in determining the statistical behaviour of the apparatus. Fermions are driven the Pauli exclusion principle, so each level must be at most single-populated and the population will obey the Fermi-Dirac statistics. In case of a non-interacting system, the wavefunction describing the ground state would be provided by a Slated determinant, accounting for the anti-symmetry required by the Pauli exclusion principle. On the contrary, for bosons there is no limit on the occupational number for each single energy level, thus in principle at T=0 all the particles lie in the ground state. Thus, in the non-interacting case, at $T=0$ the system is described by the product of the single-particle wavefunctions.

Focusing our attention to a system constituted by bosons, it is know that when the temperature gets below a certain critical temperature $T_c$ for the BEC, the ground state starts being populated and a transition to this new phase occurs. Bose-Einstein statistics takes the lead in the description of the system, accounting for purely quantum effects that are not predictable with a classical approach.

Alkali atoms are particularly suitable to produce BECs: in fact, due to their ground state electronic structure and scattering properties, they can easily be laser-cooled and trapped (\textit{cite something here}).( \textit{I'll try to read something about this in the next days, then I'll integrate})\\


\subsection{HAMILTONIAN AND TRIAL WAVEFUNCTION}
In this project we will study a population of $N$ $^{87}$Rb atoms. The hamiltonian describing such apparatus is given by
\begin{equation}
    \hat{H} = \sum_i^N \left(-\frac{\hbar^2}{2m}{\nabla}_{i}^2 +V_{ext}({\mathbf{r}}_i)\right)  + \sum_{i<j}^{N} V_{int}({\mathbf{r}}_i,{\mathbf{r}}_j)
    \label{hamiltonian}
\end{equation}
where we have used the shorthand notation for
\begin{equation*}
    \sum_{i<j}^N V_{int}({\mathbf{r}}_i,{\mathbf{r}}_j)  = \sum_{i=1}^N \sum_{j=i+1}^N V_{int}({\mathbf{r}}_i,{\mathbf{r}}_j)
\end{equation*}
The external potential describing the trap for the atoms will be modelled in two different ways, namely as a spherical (S) or elliptical (E) harmonic trap. 
\begin{equation*}
    V_{ext}(\mathbf{r}) = \Bigg\{ 
    \begin{array}{ll} \frac{1}{2}m\omega_{ho}^2(x^2 + y^2 + z^2) & (S)\\ \frac{1}{2}m[\omega_{ho}^2(x^2+y^2) + \omega_z^2z^2] & (E) 
\end{array}
\end{equation*}
For the sake of generality here we have shown the three-dimensional version of the spherical potential, but notice that this form will be adopted also to describe systems with less dimensionality. On the contrary, the elliptical potential is aimed to describe only 3D systems. The typical dimension associated to a trap for $^{87}$Rb atoms is given by $a_{ho}= \left( \hbar /m \omega_{ho}\right)^{1/2} = 1-2 \times 10^{4}$\AA. We introduce also the parameter $\gamma=\omega_z^2/\omega_{ho}^2$, which will be discussed in a moment.\\

Since we are considering a dilute system, the interaction between particles is mainly described through 2-body collisions, which in this context are well represented by the s-wave scattering length $a$ for $^{87}$Rb. This repulsive interaction is modelled by a hard-core repulsion term
\begin{equation*}
    V(\bm{r}_i, \bm{r}_j) = V(\vert \bm{r}_i - \bm{r}_j \vert ) = \Bigg\{
    \begin{array}{ll}
        \infty &  \vert \bm{r}_i - \bm{r}_j \vert \leq a \\
        0 & \vert \bm{r}_i - \bm{r}_j \vert > a 
    \end{array}
\end{equation*}
For our treatment the value of $a$ is fixed by $a/a_{ho} = 0.0043$. \\

The trial wavefunction used in the VMC approach is given by
\begin{align}
\begin{split}
    &\Psi_T(\bm{r}) = \Psi_T(\bm{r}_1, \dots, \bm{r}_N) = \\
    &= \left[ \prod_{i=1}^N g(\alpha, \beta, \bm{r}_i) \right] \left[ \prod_{j<k} f(a, \vert \bm{r}_j - \bm{r}_k \vert ) \right] \\
    &g(\alpha, \beta, \bm{r}_i) = \exp \left[ -\alpha \left( x_i^2 + y_i^2 + \beta z_i^2 \right) \right] \\
    &f(a, \vert \bm{r}_j - \bm{r}_k \vert ) = \Bigg\{ \begin{array}{ll}
        0 & \vert \bm{r}_j - \bm{r}_k \vert \leq a \\
        1 - \frac{a}{\vert \bm{r}_j - \bm{r}_k \vert} & \vert \bm{r}_j - \bm{r}_k \vert > a
    \end{array}
\end{split}
\label{wavefunctions}
\end{align}
where $\bm{r}$ is a collective variable standing for the coordinates of all the particles in the system. In principle both $\alpha$ and $\beta$ could be treated as variational parameters, however in this project only the former will be used with this connotation, while the latter will be set to a constant. Notice that $\Psi_T(\bm{r})$ written above is the trial wavefunction corresponding to the most general case treated in the project, namely interacting atoms trapped in an elliptical potential. In this case we will set $\beta = \gamma = 2.82843$. The analogue version for the non-interacting particles trapped in a spherical potential can be simply obtained from $\Psi_T$ by setting $a=0$ and $\beta=1$. In this last case, the number of terms in the quadrature sum appearing in $g(\alpha, \beta=1, \bm{r}_i)$ will be determined by the dimensionality of the system. \\

From now on, all the results will be reported as re-scaled quantities, considering $\hbar = m = \omega_{ho} = 1$.
