\subsection{NON-INTERACTING CASE}
We start now the discussion about the results obtained for systems of non-interacting particles in a spherical potential. This simple case is analytically solvable, in the sense that the analytical expressions for all the quantities involved in this project can be found. Thus in principle a VMC approach could be avoided, but we still adopted this system for the sake of testing the implemented algorithms that would have been applied in a second moment to the interacting case. 

\subsubsection*{BRUTE-FORCE METROPOLIS ALGORITHM}
It is well known that the value of the variational parameter minimizing the energy of a non-interacting system considered in this project is $\alpha_{GS}=0.5$, independently on the degrees of freedom involved in the system. This is furthermore validated by Table \ref{tab:tab_x_metropolis_analytical} and Table \ref{tab:tab_x_metropolis_numerical}, reporting data produced with the brute-force Metropolis algorithm. As previously stated, the simulations exploiting the numerical approach were performed just for a comparison in precision and CPU time with those performed using analytical formulas. One can notice that the adoption of the numerical approach for the evaluation of the second derivative appearing in Eq.\,\ref{local_energy} produces less precise results, namely $\sigma_E$ is non-null, on the contrary of what happens when the local energy is evaluated analytically. Moreover, the CPU time is also drastically increased when the numerical approach is adopted. 

\subsubsection*{ACCEPTANCE RATIO}
A more interesting aspect to notice is the behaviour of the acceptance ratio as a function of the dimensionality of the system. While using the brute-force Metropolis algorithm, the step size for generating the new proposal for each coordinate of the particles' position vector was always kept constant for all the possible tested configurations. This reflects on the acceptance ratio, which diminishes when the dimension $D$ of the apparatus grows. In fact, with the just mentioned choice for $r_{step}$ we are implicitly allowing a particle to move at each step up to a distance $r_{step} \ast \sqrt{D}$ apart from its previous position. As $D$ increases, particles can undergo larger movements and thus they are also more likely to end more distant from the origin, with a higher rejection probability for that move. \\

\subsubsection*{IMPORTANCE SAMPLING}
The considerations about precision and CPU time apply also to the simulations for $\alpha=0.5$ performed using the importance sampling algorithm, whose results are reported in Table \ref{tab:tab_x_importance_analytical} and Table \ref{tab:tab_x_importance_numerical}. The computational time spent on the single VMC runs performed here is slightly higher than the corresponding one needed for the analogous simulations employing the brute-force approach. This was expected too, since the importance sampling requires also the Greens' functions to be evaluated apart from the wavefunctions in order to discriminate between acceptance and rejection of a single move. Despite this little inconvenient, the advantages brought by this second algorithm legitimate its implementation. The bias introduced by the drift force term in the proposal for a new move drives the system to high-probability states, leading thus to a higher acceptance ratio and to a better sampling in the high-probability regions of the space of configurations, which are those that more matter for statistics. \\

\subsubsection*{STUDY OF $\alpha$-DEPENDENCE}
Once that the most basic features of our code were tested, we switched to a more information-providing case, that is $\alpha\neq 0.5$. Plots appearing respectively in Figure \ref{fig:varying_alpha_noninteract_metropolis} and Figure \ref{fig:varying_alpha_noninteract_importance} show the almost perfect overlap between the experimental points and the theoretical curve, remarking again the efficacy of VMC simulations in reproducing the expected behaviour. Furthermore, adopting $\alpha\neq0.5$ brought us away from the trivial case of null error on the estimated energy of the system. In fact, substituting $\alpha=0.5$ into Eq.\,\ref{local_energy_analytic_noninteracting}, one completely eliminates the dependence on the position of the particles. Thus, in this case $E_L$ sampled along a VMC simulation assumes always the same value. On the contrary, with $\alpha\neq0.5$ the dependence of the local energy from $\bm{R}$ is reintroduced, bringing up some more meaningful statistics in the acquired data. This was also the occasion for a first encounter with the error estimation through the blocking method, which allowed us also to access the impact of correlation in the acquired measurements. 

\subsubsection*{CHOICE OF TIME STEP $\delta t$}
We observed that the choice of the parameter $\delta t$ impacts not only on the acceptance ratio, but also on the correlation between samples. Combining data from Figure \ref{fig:dt_importance_sampling} with the column referred to $\sigma_B$ in Table \ref{tab:varying_alpha_noninteracting} helped us in tuning the parameter $\delta t$ for the successive VMC runs. We noticed that choosing a too small $\delta t$ introduced more correlation in the generated data, as can be clearly seen in Figure \ref{fig:correlation_varying_dt}. Here we see that for smaller $\delta t$ particles are less likely to be moved far away from their current position and thus more VMC cycles are needed to reach a sufficient degree of uncorrelation. For these reasons, $\delta t=0.1$ was chosen for the successive simulations, making large steps more suitable for the particles, still keeping a high acceptance ratio as typical for the importance sampling. The content of Figure \ref{fig:blocking_analysis} is reported to testify the correct behaviour of the python script used for the blocking analysis and is also a good instrument for a better comprehension of its functioning. As expected, the provided error estimate increases at each blocking iteration since always a larger fraction of the correlation term in Eq.\,\ref{err_covariance} is taken into account. After a certain number of iterations, a plateau is reached and then an erratic behaviour comes into play, due to the lack of data on which the variance is being evaluated. \\

\subsection{INTERACTING CASE}
The analysis of the non-interacting case acted as a springboard for the introduction of the interaction term between the particles. This new framework involved a system of 1, 50, 100 particles inserted in a elliptical potential and described by an asymmetric gaussian wavefunction multiplied by a Jastrow factor, as illustrated Eq.\,\ref{wavefunctions}. The investigations in this case became much less trivial, since an analytic solution for the energy of the system as a function of $\alpha$ is not available for any comparison. The simulations became also much more time demanding, due both to the complexity of the analytical formulas included in the code (e.g. Eq.\,\ref{local_energy_analitic_interacting}) and to the introduction of new computational elements which played a fundamental role in the evaluation of the needed quantities. These factors imposed a much more careful costs/benefits analysis before launching each simulation, especially a compromise had to be found between computational time and precision achieved for the results. 


\subsubsection*{STUDY OF $\alpha$-DEPENDENCE }
Table \ref{tab:varying_alpha_interacting} reports the evaluation of the energy of the system for a bunch of $\alpha$ values. The simulations involving a group of 100 particles were clearly the most demanding, thus looking at the problem from a time-saving perspective we choose to employ less steps for this kind of run. Though, the selected number of steps allowed us to keep a sufficiently low relative error on the energy estimation as a function of $\alpha$. In any case, data contained in the mentioned table provide a lot of pieces of information for what concerns the features of the new systems. In particular, one can see that the modifications added to both the hamiltonian and the wavefunction obviously changed the $\alpha$-dependence for the expected energy value. Still, the new $\alpha_{GS}$ which minimizes the energy of the system is once more included in the interval $(0.4, 0.6)$ for every analyzed $N$. This fact is proved for the case of $N=10$ by the graph of Figure \ref{fig:asymm_symm_comparison}: here we can see the great impact introduced by the modifications applied to the system with respect to the non-interacting case. The two curves show a similar shape, with the clear presence of a minimum around $\alpha=0.5$, but the energy values related to our new configurations are always higher than those corresponding to a bunch of independent particles. This behaviour is completely reasonable, since now the trap is steeper in the $z$ direction and simultaneously the inter-particles distance must be higher than the typical s-wave scattering length $a$. 

\subsubsection*{AVERAGE ENERGY PER PARTICLE}
A graphical representation of the content of Table \ref{tab:varying_alpha_interacting} is presented in Figure \ref{fig:E_over_N}. Another interesting fact is highlighted here: in the case of non-interacting particles, we experienced that the energy of the system resulted to be simply proportional to the number of particles and the dimensionality of the system, as suggested also in Eq.\,\ref{energy_analitical}. Here instead the aforementioned plot suggests that the interaction term appeared in the Hamiltonian has introduced a stronger dependence in the energy on the number of particles populating the system. This tendency is also shown in \cite{duBois}. Unfortunately the article focuses on systems populated by a higher number of particles than those that we employed, thus a more precise comparison was not affordable. Back to our case, if the interaction between the particles is switched off (i.e. one sets $a=0$) the three curves of the mentioned figure would be almost perfectly overlapping (modulo stochastic fluctuations deriving from the VMC run), while when the interaction is considered they appear as clearly distinguished. This fact is again extremely reasonable, since, considered the repulsive interaction between the particles, increasing the population of the system will lead to a larger average distance from the origin and thus to a larger average energy per particle. \\

\subsubsection*{IMPACT OF NUMBER OF PARTICLES ON $\alpha_{GS}$}
After all these preliminary considerations on the interacting system, we passed to the research of the $\alpha$ value which minimizes the energy. Results shown in Table \ref{tab:gradient_descent_noninteracting} come from the application of the gradient descent method to a non-interacting system with different number of particles: this kind of experiment was used to test the correctness of the code implementation, exploiting again the knowledge of the true value of $\alpha_{GS}$. For every case presented in the table, the descent proceeded without any issue or strange behaviour up to convergence. Once that the program was launched, the values of $\langle E_L \rangle$ described a monotonic decreasing function towards the minimum, with the corresponding derivatives with respect to $\alpha$ always decreasing in modulus down to the fixed tolerance. A much more interesting and also more demanding task is the research of this parameter in the interacting case, for which the only preliminary knowledge regards the range in which to search for it, namely $\alpha\in(0.4, 0.6)$, as suggested by Figure \ref{fig:E_over_N}. Contrarily to what happens for the non-interacting case, here we do not have at disposal a straightforward way to possibly demonstrate the convexity of $\langle E_L \rangle$ as a function of $\alpha$. However, again the shape of the curves reported in Figure \ref{fig:E_over_N} encourages us to apply the gradient descent method. 

In this much more complicated case, we had to face with a problem arising from the stochastic evaluation of the energy derivative with respect to $\alpha$. Our initial intent was to make a first descent to reach the proximity of the ground state energy and then proceed by choosing a smaller value of the parameter $\gamma$ appearing in Eq.\,\ref{alpha_k} in order to perform a finer search. However, the reduction of the learning rate revealed to be insufficient to guarantee the correct continuation of the descent, since the fluctuations in the derivative values were too large. Two different single VMC runs performed in the same conditions and with the same value for $\alpha$ could produce two estimates for $d\langle E_L \rangle / d\alpha$ close to zero, but maybe differing also in the sign. A possible solution to get rid of this inconvenient could consist in increasing the number of steps employed for each simulation along the descent, hoping that this could lead to a better estimation of the derivative. However, again we had to find a compromise between computational time and precision in the results, so we decided to proceed by implementing the gradient descent as already described in Section \ref{sec:gradient_descent_results}. The results derived from these procedures appear in Table \ref{tab:gradient_descent_interacting}: one can notice that the number of gradient descent iterations that were necessary to reach the reported results decreases as the number of particles in the system grows. This is due to the fact that for larger $N$ the magnitude of the fluctuations in the derivative increased, thus limiting the descent to a few steps. 

For all the simulations performed in the context of gradient descent, we observed the extreme sensitivity of the method to the initial guess for $\alpha$, the value of the learning rate and the number of steps performed in each MC simulation. \\


Table \ref{tab:final_GS_energy} shows that the estimated values of $\alpha_{GS}$ decrease as N increases, confirming again that adding particles to the configuration contributes to move away the system from the non-interacting approximation treated before. The simulations with the final estimates of $\alpha_{GS}$ for 10, 50 and 100 particles populating the systems were finally performed and the corresponding results are still in Table \ref{tab:final_GS_energy}. For this kind of simulations the number of steps was higher than the average amount used for the other runs treated in the project, since we aimed to reach an accurate estimation of the ground state energy of the system.

Using a even higher number of particles would have lead to a non-affordable computational time for our devices, but a result for $N=500$ was found in \cite{Nilsen2005}. In this case the value found was $\alpha_{GS}=0.475$, confirming the tendency just described for $\alpha_{GS}$.  \\

\subsubsection*{ONE-BODY DENSITY}
As a final remark, the one-body density plotted in Figure \ref{fig:one_body_density_histogram} for a bunch of $a$ values tells us about the role played by the repulsive interactions in determining their radial distribution in the system. As the minimum allowed distance between the particles increases, we can see a clear tendency to occupy positions further from the origin of the system. Here we find another evidence of what discussed above about the role played by the repulsive interaction: in principle the particles tend to occupy positions close to the origin of the system, in order to minimize its energy, but here they are forced to have reciprocal distance larger than $a$. Two of them are thus forbidden to be simultaneously close to the origin. The figure shows that an increase in the value of $a$ leads to an increase in the average distance from the origin, despite the particles have to face with a steeper potential.

The introduction of the elliptical trap caused the particles to be more confined along the $z$ direction, as shown in Figure \ref{fig:spatial_distribution_x_z}. In addition, this graph constitutes a further confirmation of the fact that adding more and more particles to the system makes it always less comparable to a non-interacting one. As a matter of fact, in this simpler case the average dispersion of the particles along a specific direction wouldn't be influenced by the number of elements appearing in the system, contrarily on what happens when the interaction in taken into account. Here we see again that a larger population leads to a greater dispersion of the particles with respect to the origin of the system. The same statements are reported also in \cite{duBois}, but here the observations are again focused on systems with thousands of particles in them. 






