In the first part of the present project we tested the implementation of the code by applying it to describe a system of non-interacting bosons in a spherical trap. Both the VMC simulations based on the analytical and numerical evaluation of the local energy for $\alpha=0.5$ provided the expected results, the latter suffering a small lack of precision and requiring a much higher CPU time. Moreover, the algorithm exploiting the analytical formula for the local energy provided excellent results also for $\alpha\neq0.5$, following the expected behaviour described by the theoretical curve of $\langle E_L \rangle$ as a function of $\alpha$. The comparison of the results coming from the brute-force Metropolis algorithm and the Metropolis-Hastings one lead us to prefer the latter because of the higher acceptance ratio and the better sampling properties provided by it. The time-step was set to $\delta t = 0.1$, which allows for having a high acceptance ratio combined with a larger average step length for the particles. This leads also to a faster exploration of the possible configurations and in a reduction of the correlation between consecutive steps. 

Switching to the interacting case, we verified that by setting the parameters $\omega_z$, $\beta$ and $a$ in order to lead back to the non-interacting system, we achieved a last successful comparison with the known analytical results. The application of the gradient descent provided us with three values for the $\alpha$ minimizing the energy of the system, respectively $\alpha_{10}=0.49751$, $\alpha_{50}=0.4891$ and $\alpha_{100}=0.482$, showing a decrease as the number of particles $N$ increases (this tendency was also confirmed in \cite{Nilsen2005}). The fluctuations in the estimate for the $\alpha$-derivative of $\langle E_L \rangle$ came into play after a few iterations of the gradient descent, preventing us to reach a higher precision in the estimate of $\alpha_{GS}$. This precision actually decreases with increasing $N$, since the magnitude of the fluctuations increases too. The MC simulations performed with the estimated $\alpha_{GS}$ provided us with $\langle E_L \rangle_{10} = 24.39843 \pm 0.00003$, $\langle E_L \rangle_{50} = 127.48 \pm 0.07$ and $\langle E_L \rangle_{100} = 266.38 \pm 0.04$, with the uncertainties estimated through the blocking method. The error affecting the result for 100 particles is surely underestimated: this may be attributable to the smaller amount of steps employed for the final simulation, not allowing for a proper estimation of the error through the blocking method. The modifications to the Hamiltonian and the introduction of the Jastrow factor in the wavefunction influenced also the behaviour of the average energy per particle introducing a dependence on $N$, contrarily to what observed for independent particles in the non-interacting case. The hard-sphere potential lead also to an increase of the average distance between the particles and the origin, this fact appearing as a little modification in the shape of the one-body density. Also in this case, the modifications became more and more evident as the population of the apparatus increased. These evidences are confirmed by the results reported in \cite{duBois}, where larger populations are analyzed.

The combination of all these observations manifests the fact that when the interaction between particles is considered, a higher $N$ will make a possible approximation to the non-interacting case always less and less precise.