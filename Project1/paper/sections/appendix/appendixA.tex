This section is devoted to show the detailed analytical derivation of the local energy formula in the general case. All the specific cases can be gotten from the final result reported below by setting properly the parameters of the system. For the sake of notation, we remind that we set $m=\hbar=\omega_{ho}=1$ and we introduce the variable $\bm{r}$ as a cumulative variable for $\{\bm{R}, \alpha, \beta\}$. The trial wavefunction describing the most general configuration for the system is reported in Eq.\,\ref{wavefunctions}, however to better face the analytical derivation we rewrite it here as:
\begin{equation*}
    \Psi_T(\mathbf{r} )=\left[ \prod_i^N g(\alpha,\beta,\mathbf{r}_i) \right] \exp{\left(\sum_{j<m}u(r_{jm})\right)}
\end{equation*} 
with 
\begin{equation}
    u(r_{ik})=\ln (f_{ik}) = \ln \left( 1-\frac{a}{r_{ik}} \right)
    \label{app:u_interaction}
\end{equation}
and
\begin{equation}
    g(\alpha,\beta,\mathbf{r}_i) = \exp{\left[-\alpha(x_i^2+y_i^2+\beta
    z_i^2)\right]}= \phi(\mathbf{r}_i)
    \label{app:gaussian}
\end{equation} 
The Hamiltonian of the system acquires the following form: 
\begin{equation*}
    \hat{H} = \frac{1}{2} \sum_i^N \left( - \nabla_{i}^2 + x_i^2 + y_i^2 +  \omega_{z}^2 z_i^2 \right) 
    +\sum_{i<j}^{N} V_{int}({\mathbf{r}}_i,{\mathbf{r}}_j)
\end{equation*}
We consider the %non trivial 
case of $r_{ij}>a,\, \forall i,j$, then  $V_{int}(\mathbf{r}_i,\mathbf{r}_j) = 0$. According to the definition of the local energy, we get
\begin{align}
    E_L(\mathbf{r}) &= \frac{1}{\Psi_T(\mathbf{r})} \hat{H} \Psi_T(\mathbf{r}) \nonumber\\
    &= -\frac{1}{2} \sum_k^N  \frac{\nabla_{k}^2\Psi_T(\mathbf{r})}{{\Psi_T(\mathbf{r})}} +\sum_k^N \frac{1}{2} \left(x_k^2 + y_k^2 + \omega_{z}^2 z_k^2 \right) 
    \label{app:localenergy_appendix}
\end{align}
One of the most nasty part of the derivation of the local energy is the evaluation of the kinetic term, which starts from the derivative of the trial wavefunction with respect to the position of the $k$-th particle: 
\begin{align}
    &\nabla_k\Psi_T(\mathbf{r}) = \nabla_k \bigg\{ \left[ \prod_i \phi(\mathbf{r}_i) \right] \exp{\left(\sum_{j<m}u(r_{jm})\right)} \bigg\} \nonumber\\
    &= \left[ \nabla_k \phi(\mathbf{r}_k) \right] \left[ \prod_{i \neq k} \phi(\mathbf{r}_i) \right] \exp{\left(\sum_{j<m}u(r_{jm})\right)} + \nonumber \\ 
    & + \left[\prod_i \phi(\mathbf{r}_i)  \right] \exp{\left(\sum_{j<m} u(r_{jm}) \right)} \nabla_k \left[ \sum_{j<m} u(r_{jm}) \right] \nonumber \\
    &= \left[ \nabla_k \phi(\mathbf{r}_k) \right] \left[ \prod_{i \neq k} \phi(\mathbf{r}_i) \right] \exp{\left(\sum_{j<m}u(r_{jm})\right)} + \nonumber\\
    & + \left[\prod_i \phi(\mathbf{r}_i) \right] \exp{\left(\sum_{j<m} u(r_{jm}) \right)} \left[ \sum_{j\neq k} \nabla_k u(r_{jk}) \right]
    \label{app:first_der_general}
\end{align}
In our specific case this yields to: 
\begin{align}
    &\nabla_k\Psi_T(\mathbf{r}) =\left[ \nabla_k  e^{  -\alpha(x_k^2 + y_k^2 + \beta z_k^2)} \right] \left[ \prod_{i \neq k}  e^{\left[ -\alpha(x_i^2 + y_i^2 + \beta z_i^2) \right]} \right] \times \nonumber \\
    & \times\left[ \prod_{j<m} \left( 1 - \frac{a}{r_{jm}} \right) \right] +  \left[\prod_i e^{ -\alpha (x_i^2 + y_i^2 + \beta z_i^2)} \right] \times \nonumber \\
    &\times \left[\prod_{j<m} \left( 1 - \frac{a}{r_{jm}} \right) \right] \left[ \sum_{j\neq k} \nabla_k \ln \left( 1 - \frac{a}{r_{jk}} \right) \right] \nonumber \\
    &= -2\alpha (x_k, y_k, \beta z_k) \Psi_T(\mathbf{r}) +  \sum_{j\neq k} \frac{1}{\left( 1 - \frac{a}{r_{jk}} \right)} \frac{a \mathbf{r}_{kj}}{r_{jk}^{3}} \Psi_T(\mathbf{r}) \nonumber \\
    &= \Psi_T(\mathbf{r}) \left[ -2\alpha (x_k, y_k, \beta z_k) + \sum_{j\neq k} \frac{a}{\left( r_{jk} - a \right) r_{jk}^2} \mathbf{r}_{kj} \right] 
    \label{app:first_der_specific}
\end{align}
To evaluate the second derivative we restart from Eq.\,\ref{app:first_der_general}.
Proceeding we get:
\begin{align*}
    &\nabla_k^2 \Psi_T(\mathbf{r}) = \left[ \nabla_k^2 \phi(\mathbf{r}_k) \right] \left[ \prod_{i \neq k} \phi(\mathbf{r}_i) \right] \exp{\left(\sum_{j<m}u(r_{jm})\right)} \\ 
    &+ \left[ \nabla_k \phi(\mathbf{r}_k) \right]  \left[ \prod_{i \neq k} \phi(\mathbf{r}_i) \right] \exp{\left(\sum_{j<m}u(r_{jm})\right)}\times \\
    & \times \left[ \sum_{j\neq k} \nabla_k u(r_{jk}) \right] + \left[ \nabla_k \phi(\mathbf{r}_k) \right] \left[ \prod_{i \neq k} \phi(\mathbf{r}_i) \right] \times \\
    & \times \exp{\left(\sum_{j<m} u(r_{jm}) \right)} \left[ \sum_{j\neq k} \nabla_k u(r_{jk}) \right] + \\ 
    &+\left[\prod_i \phi(\mathbf{r}_i) \right] \exp{\left(\sum_{j<m} u(r_{jm}) \right)} \left[ \sum_{j\neq k} \nabla_k^2 u(r_{jk}) \right] 
\end{align*}
Then dividing by $\Psi_T(\mathbf{r})$ we get: 
\begin{align*}
    &\frac{ \nabla_k^2 \Psi_T(\mathbf{r})}{\Psi_T(\mathbf{r})} = \frac{\nabla_k^2 \phi(\mathbf{r}_k)}{\phi(\mathbf{r}_k)} + 2 \frac{\nabla_k \phi(\mathbf{r}_k)}{\phi(\mathbf{r}_k)} \left[ \sum_{j\neq k} \nabla_k u(r_{jk}) \right] + \\
    & +\left[ \sum_{j\neq k} \nabla_k u(r_{jk}) \right]^2 + \left[ \sum_{j\neq k} \nabla_k^2 u(r_{jk}) \right] \\
    &= \frac{\nabla_k^2 \phi(\mathbf{r}_k)}{\phi(\mathbf{r}_k)} + 2 \frac{\nabla_k \phi(\mathbf{r}_k)}{\phi(\mathbf{r}_k)} \left[ \sum_{j\neq k} \nabla_k u(r_{jk}) \right] \nonumber\\
    & + \left[ \sum_{j\neq k} \sum_{m \neq k} \nabla_k u(r_{jk}) \nabla_k u(r_{mk}) \right] + \left[ \sum_{j\neq k} \nabla_k^2 u(r_{jk}) \right] 
\end{align*}
At this point we rewrite the gradient in spherical coordinates
\begin{equation*}
    \nabla f = \frac{\partial f}{\partial r} \frac{\mathbf{r}}{r} + \frac{1}{r} \frac{\partial f}{\partial \theta} \hat{\theta} + \frac{1}{\sin \theta} \frac{\partial f}{\partial \phi} \hat{\phi}
\end{equation*}
This choice simplifies the calculations, since the dependence of $u$ is limited to the relative distance between two particles. Applying then the gradient to $u(r_{jk})$, we get
\begin{equation*}
    \nabla_k u(r_{jk}) = \nabla_k (\mathbf{r}_k - \mathbf{r}_j) \nabla_{kj} u(r_{kj}) = \frac{\mathbf{r}_{kj}}{r_{kj}} \frac{du(r_{kj})}{dr_{kj}}
\end{equation*}
With the same reasoning, one can get also the expression for the Laplacian of $u(r_{jk})$ starting from
\begin{align*}
    \nabla^2 f &= \frac{1}{r^2} \frac{\partial}{\partial r} \left( r^2 \frac{\partial f}{\partial r} \right) + \frac{1}{r^2 \sin\theta } \frac{\partial }{\partial\theta} \left( \sin \theta \frac{\partial f}{\partial \theta} \right) + \\ 
    &+\frac{1}{r^2 \sin^2 \theta} \frac{\partial^2 f}{\partial \phi^2}
\end{align*}
which in our case reduces to
\begin{align*}
    &\nabla_k^2 u(r_{jk}) = \frac{1}{r_{jk}^2} \frac{\partial}{\partial r_{jk}} \left( r_{jk}^2 \frac{\partial u}{\partial r_{jk}} \right) \\ &= \frac{1}{r_{jk}^2}  \left( 2 r_{jk} \frac{\partial u}{\partial r_{jk}} + r_{jk}^2 \frac{\partial^2 u}{\partial r_{jk}^2} \right) = \frac{2}{r_{jk}} \frac{\partial u}{\partial r_{jk}} + \frac{\partial^2 u}{\partial r_{jk}^2}
\end{align*}
Joining all the terms evaluated up to now, we derive the expression for the kinetic energy term appearing in the expression for $E_L(\bm{r})$, namely
\begin{align}
     &\frac{ \nabla_k^2 \Psi_T(\mathbf{r})}{\Psi_T(\mathbf{r})} =  \underbrace{\frac{\nabla_k^2 \phi(\mathbf{r}_k)}{\phi(\mathbf{r}_k)}}_{\text{Term 1}} 
     + \underbrace{2 \frac{\nabla_k \phi(\mathbf{r}_k)}{\phi(\mathbf{r}_k)} \left[ \sum_{j\neq k}  \frac{\mathbf{r}_{kj}}{r_{kj}} \frac{du(r_{kj})}{dr_{kj}} \right]}_{\text{Term 2}} \nonumber\\
     & + \underbrace{\left[ \sum_{j\neq k} \sum_{m \neq k} \frac{\mathbf{r}_{kj}}{r_{kj}} \frac{du(r_{kj})}{dr_{kj}} \frac{\mathbf{r}_{km}}{r_{km}} \frac{du(r_{km})}{dr_{km}} \right]}_{\text{Term 3}} + \nonumber\\
     & + \underbrace{\left[ \sum_{j\neq k} \frac{2}{r_{jk}} \frac{\partial u(r_{jk})}{\partial r_{jk}} + \frac{\partial^2 u(r_{jk})}{\partial r_{jk}^2} \right]}_{\text{Term 4}} \nonumber \\
     \label{app:d2psi_psi_general}
\end{align}
Substituting the Eq.s\,\ref{app:u_interaction} and \ref{app:gaussian} into Eq.\,\ref{app:d2psi_psi_general} we can evaluate each term for the general case. 

\textbf{Term 1}
\begin{align*}
    \frac{\nabla_k^2 \phi(\mathbf{r}_k)}{\phi(\mathbf{r}_k)} = 2 \alpha \left[  2 \alpha (x_k^2 + y_k^2 + \beta^2 z_k^2 ) -2 -\beta \right]
\end{align*}

\textbf{Term 2}
\begin{align*}
    &2 \frac{\nabla_k \phi(\mathbf{r}_k)}{\phi(\mathbf{r}_k)} \left[ \sum_{j\neq k}  \frac{\mathbf{r}_{kj}}{r_{kj}} \frac{du(r_{kj})}{dr_{kj}} \right] = \\
    &=2 \left( -2 \alpha \left( x_k, y_k, \beta z_k \right) \right) \sum_{j\neq k}  \frac{\mathbf{r}_{kj}}{r_{kj}} \frac{a}{r_{kj} \left( r_{kj} - a \right)} \\
    &= -4 \alpha \left( x_k, y_k, \beta z_k \right) \cdot \sum_{j\neq k}  \frac{\mathbf{r}_{kj}}{r_{kj}} \frac{a}{r_{kj} \left( r_{kj} - a \right)} 
\end{align*}


\textbf{Term 3}
\begin{align*}
    &\sum_{j\neq k} \sum_{m \neq k} \frac{\mathbf{r}_{kj}}{r_{kj}} \frac{du(r_{kj})}{dr_{kj}} \frac{\mathbf{r}_{km}}{r_{km}} \frac{du(r_{km})}{dr_{km}} = \\
    &= \sum_{j\neq k} \sum_{m \neq k} \frac{\mathbf{r}_{kj}}{r_{kj}} \cdot  \frac{\mathbf{r}_{km}}{r_{km}} \frac{a}{r_{kj} \left( r_{kj} - a \right)} \frac{a}{r_{km} \left( r_{km} - a \right)} 
\end{align*}


\textbf{Term 4}
\begin{align*}
    &\sum_{j\neq k} \left[ \frac{2}{r_{jk}} \frac{\partial u(r_{jk})}{\partial r_{jk}} + \frac{\partial^2 u(r_{jk})}{\partial r_{jk}^2} \right] = \\
    &= \sum_{j\neq k} \left[ \frac{2}{r_{jk}} \frac{a}{r_{kj} \left( r_{kj} - a \right)} + \frac{a \left(a - 2 r_{kj} \right) }{r_{kj}^2 \left( r_{kj} - a \right)^2 } \right]
\end{align*}
Now, we have all the tools to evaluate analytically the local energy in Eq.\,\ref{app:localenergy_appendix}:

\begin{align*}
    &E_L(\mathbf{r})
    =-\frac{1}{2} \sum_i^N  \bigg\{ 2 \alpha \left[  2 \alpha (x_i^2 + y_i^2 + \beta^2 z_i^2 ) -2 -\beta \right] \\
    &\quad + \sum_{j\neq i} \frac{a}{r_{ij}^2 (r_{ij} - a)} \bigg\{ -4 \alpha \left( x_i, y_i, \beta z_i \right) \cdot \mathbf{r}_{ij} + 2 + \\ 
    & \quad \quad \quad + \frac{a - 2r_{ij}}{ r_{ij} - a } + \mathbf{r}_{ij} \cdot \sum_{m \neq i}   \frac{\mathbf{r}_{im}}{r_{im}} \frac{a}{r_{im} \left( r_{km} - a \right)} \bigg\} \bigg\} + \\
    & \quad\quad\quad\quad+ \frac{1}{2} 
    \sum_i^N (x_i^2 + y_i^2 + \omega_{z}^2 z_i^2) \\
    &= \alpha (2 + \beta) N -  2 \alpha^2 \sum_i^N (x_i^2 + y_i^2 + \beta^2 z_i^2 ) - \\
    &\quad \quad -\frac{1}{2} \sum_i^N \sum_{j\neq i} \frac{a}{r_{ij}^2 (r_{ij} - a)} \bigg\{ -4 \alpha \left( x_i, y_i, \beta z_i \right) \cdot \mathbf{r}_{ij} + 2 \\
    &\quad \quad \quad +\frac{a - 2r_{ij}}{ r_{ij} - a } + \mathbf{r}_{ij} \cdot \sum_{m \neq i}   \frac{\mathbf{r}_{im}}{r_{im}^2} \frac{a}{ \left( r_{km} - a \right)} \bigg\} \bigg\} \\
    &\quad \quad \quad \quad
    +\frac{1}{2} \sum_i^N (x_i^2 + y_i^2 + \omega_{z}^2 z_i^2) \\
    &= \alpha (2 + \beta) N + \sum_i^N \bigg[ (x_i^2 + y_i^2)\left(\frac{1}{2}- 2\alpha^2 \right) + \\
    & \quad + z_i^2\left( \frac{1}{2} \omega_z^2 - 2\alpha^2\beta^2 \right) \bigg] -  \frac{1}{2} \sum_i^N \sum_{j\neq i} \frac{a}{r_{ij}^2 (r_{ij} - a)} \times \\
    & \quad\quad \times \bigg\{ -4 \alpha \left( x_i, y_i, \beta z_i \right) \cdot \mathbf{r}_{ij} + 2 + \frac{a - 2r_{ij}}{r_{ij} - a } \\
    &\quad \quad \quad + \mathbf{r}_{ij} \cdot \sum_{m \neq i}   \frac{\mathbf{r}_{im}}{r_{im}^2} \frac{a}{\left( r_{km} - a \right)} \bigg\} \bigg\}
\end{align*}
which is the result reported in Eq.\,\ref{local_energy}. As previously stated, this is the most general result for the local energy in the context of this project, namely it corresponds to the the case of $N$ particles in an elliptical potential. However, by simply imposing conditions on $a$, $\beta$ and $\omega_z$ and possibly reducing the dimensionality of the system, every other treated case can be restored.