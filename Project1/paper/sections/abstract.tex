\begin{abstract}
In this project we implemented a Variational Monte Carlo (VMC) code to reproduce the properties of a Bose-Einstein condensate (BEC) \cite{einstein_original}, at first considering the bosons as independent and later adding a pair interaction represented by a hard-sphere potential. Both the brute-force Metropolis \cite{metropolis} and the Metropolis-Hastings algorithms \cite{hastings} were tested while considering the non-interacting case, revealing the better performance provided by the latter in terms of higher acceptance ratio and better sampling properties. The energy of the system was evaluated on varying the number of particles, the dimensionality and the variational parameter $\alpha$, revealing the accordance with analytical solutions available for this simple case. A statistical analysis of the results was performed through the blocking technique \cite{Marius}, which provided us with estimate of the degree of correlation between data generated within a VMC simulation. 

The interaction between the particles was then taken into account, along with a modification of the trap shape. The research performed through the gradient descent method revealed that the value of $\alpha$ minimizing the energy of the system decreases when the number of particles grows. This shows that adding elements into the apparatus moves it further and further away from the ideal condition described by the non-interacting model. This fact is supported by both an increased average energy per particle and an increased average distance from the origin when more particles are included in the system. Finally, the evaluation of the one-body density revealed also a reduction in the displacement along the direction in which the trap has been made steeper.
\end{abstract}