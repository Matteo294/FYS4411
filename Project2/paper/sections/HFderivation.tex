In this section we want to provide a derivation of the Hartree-Fock equations mainly referring to \cite{bransden}, in the special case of the ground state of an atom having $N$ electrons. \\
The hamiltonian that describes such a system is assumed to be 
\begin{equation*}
    \mathcal{H} = \mathcal{H}_1 + \mathcal{H}_2
\end{equation*}
where $\mathcal{H}_1$ is the single particle hamiltonian (or one body hamiltonian)
\begin{equation*}
    \mathcal{H}_1 = \sum_{i=1}^N h_i = \sum_{i=1}^N -\frac{1}{2}\nabla_{\ve{r}_i}^2 + V(r_i)
\end{equation*}
and $\mathcal{H}_2$ is the interaction term
\begin{equation*}
    \mathcal{H}_2 = \sum_{i, j = 1}^N \frac{1}{r_{ij}}
\end{equation*}
Since we are dealing with a system of fermions, the wavefunction describing the ground state of the system $\Phi^*(q_1, \dots, q_N)$ must be antisymmetric under the action of the permutation operator $P$ so that
\begin{gather*}
    P_{ij} \Phi^*(q_1, \dots,q_i, \dots, q_j, \dots, q_N) = \\
    = \Phi^*(q_1, \dots, q_j, \dots, q_i, \dots, q_N) = \\ 
    = - \Phi^*(q_1, \dots, q_i, \dots, q_j, \dots, q_N) 
\end{gather*}
The central idea of the method consists in assuming that the ground state wavefunction $\Phi^*(q) \equiv \Phi^*(q_1, \dots, q_N)$ can be approximated by a Slater determinant of single particle wavefunctions
\begin{equation*}
    \Phi^*(q) \approx \Phi(q) \equiv \frac{1}{\sqrt{N!}} \ \begin{vmatrix} \phi_{\alpha}(q_1) \ \phi_{\beta}(q_1) \ \dots \ \phi_{\nu}(q_1) \\ 
    \phi_{\alpha}(q_2) \ \phi_{\beta}(q_2) \ \dots \ \phi_{\nu}(q_2) \\
    \dots \\
    \dots \\
    \dots \\
    \phi_{\alpha}(q_N) \ \phi_{\beta}(q_N) \ \dots \ \phi_{\nu}(q_N)
    \end{vmatrix}
\end{equation*}
and then make use of the variational principle
\begin{equation}
    \bracketOP{\Phi^*}{\Phi^*}{H} = \mathbb{E}[\Phi^*] \leq \mathbb{E}[\Phi] = \bracketOP{\Phi}{\Phi}{H}
    \label{eq:en_ground_slater_1}
\end{equation}
to find the single particle function $\phi_{\alpha}, \dots \phi_{\beta}$ that minimize the expectation value of the energy $\mathbb{E}[\Phi]$. \\
Each of the subscripts $\alpha, \beta, \dots, \nu$ used in the definition of $\Phi$ refer to a particular set of quantum numbers $(n, l, m_l, m_s)$ so that the notation $\phi_{\mu}(q_i)$ stands for the single electron orbital identified by the set of quantum numbers $\mu = (n^\mu, l^\mu, m_l^\mu, m_s^\mu)$ evaluated in the position of the particle $i$. \\
We also require that ${\phi_{\mu}(q_i)}$ is an orthonormal basis set, so that $\langle \phi_{\mu}|\phi_{\lambda}\rangle =\delta_{\lambda, \mu} $ .\\
The approximated wavefunction can be rewritten by making use of the antisymmetrizer operator $\mathcal{A}$:\footnote{The definition and all the properties of the antisymmetrizer operator can be found at \href{https://en.wikipedia.org/wiki/Antisymmetrizer}{https://en.wikipedia.org/wiki/Antisymmetrizer}}

\begin{align*}
    \Phi(q) &= \frac{1}{\sqrt{N!}}\bigg(\sum_P (-1)^P \hat{P} \bigg) \Phi_H(q) \\
     &= \sqrt{N!} \, \mathcal{A} \, \Phi_H(q)
\end{align*}
where $\hat{P}$ indicates the permutation operator and $\Phi_H(q)$ is:
\begin{equation*}
    \Phi_H(q) = \phi_\alpha(q_1) \, \phi_\beta(q_2) \, \dots \, \phi_\nu(q_N)  
\end{equation*}
Using the variational principle one can estimate the energy of the ground state with the initial guess on $\Phi$.
Starting from equation \ref{eq:en_ground_slater_1}
\begin{align*}
    \mathbb{E}[\Phi] &= \bracketOP{\Phi}{\Phi}{H} \\
    &= \bracketOP{\Phi}{\Phi}{H_1} + \bracketOP{\Phi}{\Phi}{H_2}
\end{align*}
Let us study the $H_1$ term: 
\begin{align*}
    \bracketOP{\Phi}{\Phi}{H_1} &= N! \bracketOP{\Phi_H}{\Phi_H}{\mathcal{A}H_1\mathcal{A}} \\
    &= N! \bracketOP{\Phi_H}{\Phi_H}{H_1\mathcal{A}^2} \\
    &= N! \bracketOP{\Phi_H}{\Phi_H}{H_1\mathcal{A}} \\
    &= \sum_{i=1}^N \sum_P (-1)^P \bracketOP{\Phi_H}{\Phi_H}{h_i\hat{P}} \\
\end{align*}
In the last sum the only term that survive is when the permutation operator coincides with the identity operator. This is because we choose an othonormal basis set for $\psi_{\mu}(q)$, this yields to: 
\begin{align*}
    &= \sum_{i=1}^N  \bracketOP{\Phi_H}{\Phi_H}{h_i}
\end{align*}
Now one can rewrite the summation running over the N individual quantum states rather than particle. 
This notation will be helpful in the derivation. 
\begin{align*}
    &= \sum_{\lambda} \bracketOP{\phi}{\phi}{h_i}\\
    &= \sum_{\lambda} I_\lambda
\end{align*}
Where $I_{\lambda}$ is the average value of the individual hamiltonian $h_i$ relative to the spin orbital $\phi_{\lambda}$.
Now let's focusing on the $H_2$ interaction term: 
\begin{align*}
    \bracketOP{\Phi}{\Phi}{H_2} &= N! \bracketOP{\Phi_H}{\Phi_H}{H_2 \mathcal{A}} \\
    &= \sum_P (-1)^P  \bracketOP{\Phi_H}{\Phi_H}{H_2 P} \\
    &= \sum_{i<j} \sum_P (-1)^P  \bracketOP{\Phi_H}{\Phi_H}{\frac{1}{r_{ij}} P}
\end{align*}
where $r_{ij}=|r_i -r_j|$. Her the only terms that are non zero are those where the operator $P$ coincides with the identity or when $P$ exchanges the index $i \leftrightarrow j$:   
\begin{align*}
    &= \sum_{i<j} \bracketOP{\Phi_H}{\Phi_H}{\frac{1}{r_{ij}}(1-P_{ij})}
\end{align*}



