Both the general and restricted implementations allowed us to reproduce the results for the ground state energy and one-body density reported in \cite{Zanghellini_2004}. The analysis of the convergence properties for the two spin representations revealed that the general solver allowed the system to explore regions even lower in energy, but this would have prevented us from further constructive comparisons with the results obtained by Zanghellini et al. \\

The so-obtained matrix coefficient generated after the convergence of the general solver was then used as seed for all the further time evolutions. In particular, the chosen integrator allowed us to reproduce again the results presented in \cite{Zanghellini_2004} for what concerns the overlap integral, enforcing the reliability of our code. Other simulations performed also with different values of $\omega$ revealed that both the curves $\xi(t)$ and $\overline{x}(t)$ show an approximate periodicity given by $2\pi/\Omega$, which appears more evident only for small $t$ and $\Omega \gtrsim 8\omega$. In particular, the higher the frequency of the laser with respect to the one associated to the harmonic potential, the longer is the time spent by the system near to the initial ground state. This behaviour may be symptom of the fact that the system shows a certain inertia which prevents it from fast adaptations to the changes in the potential induced by the laser field, especially when those changes occur at high frequency. Lower values of $\omega$ put the system closer to a resonant behaviour, thus letting the system to adapt to the induced changes with the consequent early breaking of the mentioned symmetries. \\

The time-evolution with the laser source first active and then switched off was then performed. The spectrum associated to $\xi_T(t)$ was populated by many transition energy values: their spacing reminds to the typical spectrum associated to a system trapped in a harmonic oscillator potential, but further analysis should be performed in this context. The Fourier analysis of the curve $\overline{x}(t)$ for $t>T$ revealed that not all the expected frequencies appeared in the plot: this may be attributable to the annihilation of some components given by a null value of $\bracketOP{\Psi_i}{\Psi_j}{\hat{x}}$. 

Many more configurations could be tested for each of the mentioned points, testing also different values for many parameters.



